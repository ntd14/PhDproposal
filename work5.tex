One of the major differences between the model presented here and in previous
literature is the inclusion of fibres intertwining macrofibrils. Recently Chang
et al. (2014) measured the pore size and shape within tension wood and oppressed
wood of poplar during cell wall maturation. With this recent advancement,
reasonable assumptions around how regularly fibrils interact with other fibrils
outside of their host macrofibril can be made. It is thought that these pores
occur between joining fibrils connecting the macrofibrils into the larger
structure that is the forming cell wall. If the deposition of lignin into the
pores forcing the fibrils apart is the mechanism by which growth stresses
develop the quantity of pores and pore sizes are important parameters to
investigate as they will largely affect the ability of the mechanism to cause
stress.