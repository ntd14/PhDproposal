Growth stresses are the tendency for cells to contract longitudinally and expand tangentially while developing their secondary cell wall. It is thought these stresses develop in normal wood in order to increase mechanical stability due to the increase in safety factor of compressional failure they provide \cite{mattheck1997wood}. There are two independent hypothesis as to how growth stresses form. The lignin swelling hypothesis --boyd 1950-- argues that the cellulose fibril aggregates in the cell wall are pushed apart tangentially by the deposition of lignin into the secondary cell wall, resulting in the dimensional change. Cellulose contraction --bamber 1978? and 2000ish--- claims the cellulose chains change their length resulting in the cellular distortion. A merging of the two has also been proposed and shown to fit experimental data well --unified papers--. Regardless of the mechanism the result is a longitudinal contraction and a tangential expansion of the cells, resulting in the stem periphery being under tension and the pith under compression \cite{Archer_1987}. 