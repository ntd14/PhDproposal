Growth stresses in normal wood result from the tendency for cells to contract longitudinally and expand tangentially while developing their secondary cell wall. It is thought these stresses develop in normal wood in order to increase mechanical safety by improving compressional strength \citep{mattheck1997wood}. There are two independent hypotheses as to how growth stresses form. The lignin swelling hypothesis \citep{ISI:A1950XU10300003} argues that the cellulose fibril aggregates in the cell wall are pushed apart tangentially by the deposition of lignin into the secondary cell wall. Cellulose contraction \citep{Bamber1979,bamber2001general} claims the cellulose chains change their length resulting in the cellular distortion. A merging of the two has also been proposed and shown to fit experimental data well \citep{okuyama1986,Okuyama_1994,yamamoto1991,ISI:A1992HP18200001, Yamamoto_1998}. Regardless of the mechanism the result is a longitudinal contraction and a tangential expansion of the cells, resulting in the stem periphery being under tension and the pith under compression \citep{Archer_1987}. 