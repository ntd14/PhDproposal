Wardrop (1965) commented that a tensile stress generated in the cellulose
transitioning into a crystalline state could be the explanation for cells
contracting during the formation of the secondary wall. Cellulose contraction
aligned well with the observation of the G-layer (which has  a very low
MFA) being common in a number of tension wood producing species, and also gave
the ability for low MFA normal wood to contract. Bamber (1978) further argued
cellulose contraction claiming turgor pressure in normal wood cells remained
high enough that the cells did not contract before the lignin was deposited,
once/during lignin deposition the cellulose became crystalline and shrunk,
causing the cell to become shorter, the mechanism for tension wood is
essentially the same. Compression wood on the other had was explained by the
cellulose being laid down and then the turgor pressure decreasing, causing the
cell to contract before lignin was deposited. In turn the cellulose was under
compression, resulting in the tendency for the compression wood cells to
expand.