Other mathematical techneques have also been applied to plant cell walls, \cite{HEPWORTH_1998} used a descrete element aproach, with limited results. Hydrogen bond dominated solids models have been used to descibe paper \cite{nissan1997link}\cite{batten1987unified}\cite{nissan1987unified}\cite{batten1987unified}. \cite{Zhan_2014} used a representative volume element method to describe hardwood, however the resolution is above that of the cell wall.

Recently molecular dynamics methods have been used to simulate small volumes of
the cell wall in order to investigate the nano structure which may be present \cite{Charlier_2012}\cite{Zhang_2009}\cite{Sangha_2011}\cite{houtman1995cellulose}.

AFM and electron microscopy have been used to probe the cell wall at the level fibril aggregates, showing that the fibrils are not straight and instead meander through the cell wall in a general direction. The fibril aggregates join and separate creating a distribution of pore sizes and shapes. The incorporation of the fibril aggregates architecture has yet to be incorporated into cellular models. 