The second theory put forward in an attempt to correct the issues surrounding
cellulose lengthening during crystallisation 
hemicelluloses form within the fibrils and push them apart causing the
cellulose fibrils to contract. Interestingly mechanically this is very similar
to the lignin swelling hypothesis. By causing the MFs to no longer run straight,
instead they have to use some of their length to deviate passed a cluster of
hemicelluloses consequently shortening the over all distance the fibril can
cover. One side effect of having these deviations is fibrils should not have a
consistent cross sectional area over their whole length, where the
hemicelluloses have been deposited should result in an increased cross section.