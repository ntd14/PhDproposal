\textbf{Results and Discussion}
This scoping study was substantially larger than any preceding investigations, assessing growth-strain and other wood properties on more than 600 Eucalyptus bosistoana and other eucalypt species at an early age (less than 2 years old). The results from these trials showed that growth-strain is heritable, and family rankings varied little whether grown from seed or coppiced from existing root systems. Tab. 1 shows the family mean Spearman rank coefficients of the tested wood properties whether grown from seed or coppice. Fig. 2 shows 20 families (8 half-sib replicates each) ordered by median growth-strain. The family rankings were similar (Spearman coefficient of ~0.77). In particular the top 3rd of the families, i.e. those with the lowest growth-strain were the best in both trials (Fig. 2). Growth-strain increased after coppicing. When plants are coppiced from existing root systems they emerge from the side of the old trunk resulting in a hockey-stick shaped lower stem. Given the nature of the testing procedure, it is suspected that the increase in growth-strain with coppicing was due to the formation of tension wood rather than an indicator that older trees will possess significantly higher growth-strain. Analysis revealed a narrow sense heritability (h\textsuperscript{2}) of 0.44 with a 95\% credible interval of the posterior distribution from 0.20 to 0.67. Tab. 2 shows the h\textsuperscript{2} values for the tested wood properties.
