 The first attempt at mathematically defining a fibre or tracheid was a single layer two phase composite model consisting of the S2 layer composed of aligned cellulose fibrils and isotropic lignin \citep{Barber_1964}. This model was quickly improved on by \citet{mark1967cell} and \citet{Cave_1968} using continuum mechanics methods. \citet{mark1967cell} provides an in depth discussion concerning both experimental and theoretical estimation of tracheids mechanical properties. \citet{Cave_1968} developed the model to include a gaussian distribution of the MFA. \citet{bergander2002cell} developed a nine layer model which emphasised the importance of the inclusion of the S1 and S3 layers when estimating transverse elastic properties. \citet{harrington2002hierarchical} developed these ideas to incorporate a three stage homogenization procedure utilizing nanostructural (supramolecular), ultrastructural (cell wall) and microstructural (whole cell) scales in order to estimate a number of material properties of softwood. Further small advancements have been made to the model over the last decade, for the most recent see \citet{Sun_2014,Saavedra_Flores_2014,wang2013gradual} or \citet{faisal2013multiscale}.
 
 

 
