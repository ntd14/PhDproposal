 The first attempt at mathematically defining a fibre or tracheid was a single layer two phase composite model consisting of the S2 layer composed of aligned cellulose fibrils and isotropic lignin \cite{Barber_1964}. This model was quickly improved on by \cite{mark1967cell} and \cite{Cave_1968} using continuum mechanics methods. \cite{mark1967cell} provides a very in depth discussion concerning both experimental and theoretical estimation of tracheids mechanical properties. \cite{Cave_1968} developed the model to include a gaussian distribution of the MFA. \cite{bergander2002cell} developed a nine layer model which emphasised the importance of the inclusion of the S1 and S3 layers when estimating transverse elastic properties. \cite{harrington2002hierarchical} developed these ideas to incorporate a three stage homogenization procedure utilizing nanostructural (supramolecular), ultrastructural (cell wall) and microstructural (whole cell) scales in order to estimate a number of material properties of softwood.
 
 

 
