Another outstanding issue, common to many biological problems is why do
particular traits vary so much between individual and species? One of the
more debated topics around growth stress generation is whether or not the generation
mechanisms for stress in reaction wood are extreme versions of the same
mechanisms in normal wood. The G-layer is not found in normal wood, however not
all tension wood producing species produce G-layers. Lignin swelling could
potentially fit this criteria for normal and compression wood, however
modification of Boyds theory would be needed address the dependence of a MFA
as some wood with lower than 40 degree MFA still produces compressive forces.
There has been reported to be little lignin within the G-layer, which is
suspected to be responsible or at lest partly responsible for tension
generation. Boyds theory combined with excessive mild compression wood
formation in core wood still allows for the same tension generation mechanisms
to be used by older cambiums, as long as the MFA is suited to the task.