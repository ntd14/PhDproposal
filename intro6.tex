Primarily this work focuses on fibres as they are the structural cells expected
to be responsible for growth stresses in normal and reaction wood within
hardwoods --ref--. Fibres consist of a number of cell wall layers depending on
the species, the particular cell and its primary function. Normal wood fibres
within Eucalyptus species consist of a middle laminar (connecting the fibre to
the sounding cells) a primary cell wall and a secondary cell wall consisting of
S1, S2 and S3 layers (produced in chronological order so the exact composition
will change depending on the cells developmental stage)--ref--. The S2 layer is
the largest layer and consists of cellulose macrofibrils wrapped helically
around the cells longitudinal axis --ref--. Cellulose is contained within a
matrix of hemicelluloses (examples) and lignins giving the cell wall properties
of a fibre reinforced matrix --ref--.