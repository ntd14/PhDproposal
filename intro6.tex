Primarily this work focuses on fibres as they are the structural cells expected
to be responsible for growth stresses in normal and reaction wood within
hardwoods \cite{archer1987growth}. Fibres consist of a number of cell wall layers depending on
the species, the particular cell and its primary function. Normal wood fibres
consist of a middle laminar (ML, connecting the fibre to
the sounding cells) a primary cell wall (P) and a secondary cell wall (S) consisting of
S1 and S2 (and sometimes S3) layers (produced in chronological order so the exact composition
will change depending on the cells developmental stage)\cite{barnett1981xylem}. The S2 layer is
the largest layer and consists of cellulose macrofibrils wrapped helically
around the cells longitudinal axis. The cellulose is contained within a
matrix of hemicelluloses and lignins giving the cell wall properties
of a fibre reinforced matrix \cite{niklas2012plant}.