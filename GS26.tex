The critical MFA was determined to be between
25 and 30 degrees for a number of species --refs--, Okuyama et al 1993 (growth
stresses in tree, in Japanese) and \cite{ISI:A1995QJ03000001} suggested the
unified hypothesis. Although the idea of both lignin swelling and cellulose
contraction being responsible for growth stress development had been suggested
before --refs-- it was formalised here. In an attempt to solve the critical MFA
discrepancy \cite{ISI:A1995QJ03000001} augmented the Barber and meylan --ref-- cell wall
model to include a S1 layer. The resulting model was the first to be able to
account for generation of both tensile and compressive stresses over a wide
range of MFAs, however this was only achievable using unnatural parameter values, (in
particular --what were they--). The S1 layer introduced utilizes a constant MFA
of 90 degrees, with the S2 layer varying from 0$^{\circ}$ to 60$^{\circ}$. Cell wall maturation
occurred in two discrete steps, first the cellulose framework is constructed then
the lignin deposition occurs. From the model they showed that with an
increasing S1 layer thickness the critical MFA reduces. They found the model
was unable to produce realistic tangential strains unless unnatural parameters
were used.