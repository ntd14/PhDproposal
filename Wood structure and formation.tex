\section{Wood structure and formation}
Softwoods have a simpler microstructure than hardwoods, consisting mainly of
axially elongated tracheids which provide both mechanical support and water
transport \cite{bowyer2007forest}. Hardwoods contain a more complex micro structure with a number
of different cell types. Fibres, similar to softwood tracheids provide
structural support however it is their primary function, with vessels providing
conduction. Vessels are comprised of
multiple vessel elements being joined at the ends to form long conduits, which
can extend short distances (often less than 200mm) or can be as long as the
height of the tree \cite{walker1993primary}. Rays are formed from radially orientated cells often tracheids or parenchyma. Rays provide a mechanical advantage by diverting the axial force flow reducing buckling and shear stresses between fibres \cite{mattheck1997wood}. Many further cell types and functions exist, but more detailed wood anatomy and
has little bearing on this project and is discussed in a number of wood anatomy
texts \cite{fromm2013cellular}.