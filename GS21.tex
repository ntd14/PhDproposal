A common argument that is made for the cellulose contraction hypothesis is the
correlation between cellulose content and strain. Higher proportions of
cellulose compared to lignin correlate to tensile strains \citep{Sugiyama_1993,Qiu_2008,Yang_2006}, while high lignin
content correlates well to compressive strains \citep{ISI:A1991FD97000001,Okuyama_1998}. It has been well
reported that compression wood is partly characterised by an increase in
lignin content \citep{timell1986compression}, which has been used as an argument for the lignin
swelling hypothesis. Tension wood however is often but not necessarily
correlated with an increased proportion of cellulose.% --tests on tension wood with no G-layer-- 
Within tension wood of G-layer producing species tensile
strain and whole cell cellulose content correlate well due to the G-layer
having a very low lignin content \citep{gardiner2014biology}. The proportions of cellulose and lignin
within the cell after the G-layer has been removed do not share this correlation. \citet{timell1969chemical} found a higher concentration of lignin within the S2 layer than in normal wood when the G-layer was present. 
