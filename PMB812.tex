Over the last decade, the New Zealand Drylands Forest Initiative (NZDFI) has obtained the largest collection of seed in the world for a number of naturally durable eucalypt species, including E. bosistoana, with the aim of establishing a fast-growing, naturally durable, super-stiff, sustainable plantation timber resource in New Zealand. The basis is a breeding programme which gives wood properties the same priority as growth, form and tree health. This novel approach to tree improvement also includes very early screening (age 1-2) to ensure a timely deployment of improved germplasm \citep{altaner_developing2015}. With the “Splitting” test, screening of the entire genetic stock is now a practical solution to remove growth-strain induced wood defects. A 10,000 tree trial consisting of ~200 families each with 50 half-sibling replicates of E. bosistoana has been established. The trial will be harvested at an age between 18 and 24 months (late 2016 - 2017) and evaluated for growth-strain, as well as improved for early form, growth, stiffness, volumetric shrinkage and basic density. As the tests are destructive the superior individuals need to be rescued by coppicing. Propagation of coppice cuttings is also providing a fast route to deploy improved material to the forestry sector. Additionally, a number of long-term field trials have been established (as early as 2009) throughout New Zealand to provide longer term studies of wood properties in particular heartwood formation. 