\section{Results to date}
\subsection{PMB8 Procedings}
\subsection{Screening eucalypts for growth-strain}
Nicholas T. Davies, Monika Sharma, Clemens M. Altaner and Luis A. Apiolaza
New Zealand School of Forestry, University of Canterbury, New Zealand
ntd14@uclive.ac.nz 
\subsubsection{Introduction}
Eucalypt species are fast-growing and can produce high quality timber for appearance and structural products including Laminated Veneer Lumber (LVL). Eucalypts can contain large growth-strains which are associated with log splitting, warp, collapse and brittleheart. These impose substantial costs on processing (Yamamoto 2007). Costly, and only partially effective, mitigation strategies have been developed to reduce wood defects induced by growth-strain. As growth-strain is highly heritable, an alternative approach is to select and grow individuals which display low growth-strain. Until now measurement of growth-strain has been difficult, time consuming and expensive, preventing the assessment of the large number of trees needed by a breeding programme (Altaner 2015). As an example, the largest sample number in any reported growth-strain study was smaller than 230 trees (Solorzano Naranjo 2011). Traditionally selections are made when trees are older, not only increasing costs (e.g. trial management, sample handling) but also substantially extending the breeding cycle and delaying the deployment of improved germplasm (Altaner 2015). Developments at the University of Canterbury have resulted in a unique growth-strain measurement method supported by theoretical analysis (Entwistle 2014) - dubbed the “Splitting” test. It allows for rapid growth-strain assessment on young trees (Chauhan 2010).

\subsubsection{Methods}
The development of the splitting test, based on the pairing test (Chauhan 2010), has resulted in a simple and quick method of growth-strain measurement which can be used on small stems. The test relies on the release of stored strain energy in a single plane via cutting through the pith as can be seen in Fig. 1. Chauhan (2010) derived Equation 1 to estimate the deflection given growth-strain from the sample geometry.


  