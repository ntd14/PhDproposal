During primary wall formation rapid elongation occurs. When the cells divide
from their parents they remain fixed to their neighbors via the ML
--ref--. The internal hydrostatic (turgor) pressure causes cell expansion
--ref--. The osmotic flow of water from the outside the cell to the inside which
is constrained by the primary cell wall causes increasing tension --ref--.
Because the centre of the cell has restricted movement, in order for elongation
(to dissipate the increasing tensile forces generated from the inflow of water)
to occur the cell turns the biosynthesis of cell wall constituents to produce
tip growth --ref--. Growth at the tips of the cells allows for the cells to
remain a constant thickness, so no stretching is needed during the elongation
phase.