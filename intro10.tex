During primary wall formation rapid elongation occurs. The internal hydrostatic (turgor) pressure causes cell expansion controlled by the orientation of the cellulose microfibrils \cite{Tyerman_2002} \cite{16261190}.
Because the centre of the cell has restricted movement, elongation
(to dissipate the increasing tensile forces from the turgor pressure)
occurs by
tip growth \cite{taiz2006plant}. Growth at the tips of the cells allows for the cell walls to
remain at a constant thickness, so no or minimal stretching is needed during the elongation
phase.