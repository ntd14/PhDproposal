During primary wall formation rapid elongation occurs. The internal hydrostatic (turgor) pressure causes cell expansion controlled by orientation of the cellulose microfibrils \cite{Tyerman_2002} \cite{16261190}.
Because the centre of the cell has restricted movement, in order for elongation
(to dissipate the increasing tensile forces from the turgor pressure)
to occur the cell turns the biosynthesis of cell wall constituents to produce
tip growth --ref--. Growth at the tips of the cells allows for the cells to
remain a constant thickness, so no or minimal stretching is needed during the elongation
phase.