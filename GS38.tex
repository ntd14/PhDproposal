It is worth noting that because bark is also formed by cambial cells
differentiating, when the cambium divides to the outside, the cells (typically)
phloem --check-- become bark. The bark is therefore under transverse tension.
Some of this is alevated via the bark peeling and lateral ray expansion as it ages and is forced further
from the cambium, however the remaining ring can still be providing a
significant stress on the stem as it tries to contract. Bark rings contracting
when removed from disks has been observed by kraus 1867 Krabbe 1882. Okuyama et
al 1981 measured 750 micro strains in the bark of Japanese cedar. 