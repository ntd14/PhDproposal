As reviewed by \cite{kubler_1987}, \cite{Okuyama_1981} adopted the use of strain gauges to measure stem surface
stresses of particular layers of wood. Other methods were also derived around
the same time, \cite{gueneau1973}\cite{gueneau1973b} and \cite{kikata1977} investigated drilling holes near strain gauges to release strains.
\cite{Gueneau1974} and \cite{Saurat_1976} introduced an apparatus which utilised two knife
blades at a set distance, one knife blade bent as the strain was released via
drilling the strain release was measured on the blade. 

In an attempt to introduce a rapid measurement for growth stress screening on young trees \cite{Chauhan_2010} and \cite{Entwistle_2014} introduced and tested a variant of the pairing test. \cite{naranjo2012early} and \cite{Aggarwal_2013} have used the test for investigating genetic relationships within \textit{Tectona grandis} and \textit{Eucalyptus tereticornis} clones. \cite{Chauhan_2011} used the test during an investigation of juvanile \textit{Eucalyptus regnans} tension wood properties. The test in its current form involves taking a section of stem 300mm long and splitting it 200mm though the pith from the big end the opening of the two big end half rounds is then measured (unpublished, updated from \cite{Chauhan_2011}).  