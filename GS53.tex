The commonly accepted argument for the reason of growth stress existence is
the mechanical hypothesis. The mechanical hypothesis argues that a number of wood
properties, including the development of growth stresses evolved in order to
provide increase mechanical stability of trees to improve their
survival rate. The mechanical hypothesis as applied to growth stresses argues,
because wood is stronger in tension than in compression, by preloading the outer
edge of the stem in tension the non-destructive bending radius on
the inside of the curve is increased when a force is bending the stem \cite{barnett2003wood}.