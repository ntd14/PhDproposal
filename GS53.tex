The commonly accepted argument for the reason of growth stresses existence is
the mechanical hypothesis. The mechanical hypothesis argues that a number of wood
properties, including the development of growth stresses evolved in order to
provide increase mechanical stability of trees in order to increase their
survival. The mechanical hypothesis as applied to growth stresses argues that
because wood is stronger in tension than compression by preloading the outer
edge of the stem in tension it increases the non-destructive bending radius on
the inside of the curve when a force is applied causing the stem to bend.