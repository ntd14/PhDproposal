In an attempt to introduce a rapid measurement for growth stress screening on young trees \cite{Chauhan_2010} and \cite{Entwistle_2014} introduced and tested a variant of the pairing test. \cite{naranjo2012early} and \cite{Aggarwal_2013} have used the test for investigating genetic relationships within \textit{Tectona grandis} and \textit{Eucalyptus tereticornis} clones. \cite{Chauhan_2011} used the test during an investigation of juvenile \textit{Eucalyptus regnans} tension wood properties. The test in its current form involves taking a section of stem ~300 mm long and splitting it ~200 mm though the pith from the big end. The opening between the two big end half rounds is then measured (unpublished, updated from \cite{Chauhan_2011}).  