\subsubsection*{Introduction}
Eucalypt species are fast-growing and can produce high quality timber for appearance and structural products including Laminated Veneer Lumber (LVL). Eucalypts can contain large growth-strains which are associated with log splitting, warp, collapse and brittleheart. These impose substantial costs on processing \citep{yamamoto2007slides}. Costly, and only partially effective, mitigation strategies have been developed to reduce wood defects induced by growth-strain. As growth-strain is highly heritable, an alternative approach is to select and grow individuals which display low growth-strain. Until now measurement of growth-strain has been difficult, time consuming and expensive, preventing the assessment of the large number of trees needed by a breeding programme (Altaner 2015). As an example, the largest sample number in any reported growth-strain study was smaller than 230 trees (Solorzano Naranjo 2011). Traditionally selections are made when trees are older, not only increasing costs (e.g. trial management, sample handling) but also substantially extending the breeding cycle and delaying the deployment of improved germplasm (Altaner 2015). Developments at the University of Canterbury have resulted in a unique growth-strain measurement method supported by theoretical analysis (Entwistle 2014) - dubbed the “Splitting” test. It allows for rapid growth-strain assessment on young trees (Chauhan 2010).

