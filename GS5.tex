Stresses relating to reaction wood received more attention through the 30s and
40s for both soft and hardwoods. Jacobs (1945) stated that the reorientation of
stems is caused by a modification to the already existing stress gradient
throughout the stem. One option he presented was simply that the eccentric
growth causes larger number of cell sheaves to be added to the upper side of the
curve each providing the same amount of contraction force, this results in a
angle correction even with identical cells. Sap tension was also considered, but
more importantly Jacobs notes the possibility of tensions being formed within the
cell walls of tension wood.
Munch 1938 speculated that the addition of matter into the cell wall could cause
compression wood. ..
Jacobs 1945 also found that it was commonly the case that the amount of
compression wood developed and the stem angle recovery had a poor relationship.
He suggested maybe it was the normal strain pattern in tension which
correct the lean, the compression wood mealy acted as a pivot, not contributing
a tensile force on the lower side of the stem.
