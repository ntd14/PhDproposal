Stresses relating to reaction wood received more attention through the 30s and
40s for both soft and hardwoods. \citet{jacobs1945l} stated that the reorientation of
stems is caused by a modification of the already existing stress gradient
throughout the stem. One option he presented was simply that the eccentric
growth causes larger numbers of cells to be added to the inner side of the
curve. Each cell providing the same amount of contraction force results in a
angle correction. Sap tension was also considered, but
more importantly Jacobs notes the possibility of tensions being formed within the
cell walls of tension wood.
\cite{jacobs1945l} also found that that the amount of
compression wood developed and the stem angle recovery had a poor relationship.
He suggested maybe it was the normal strain pattern in tension which
correct the lean, the compression wood mealy acted as a pivot, not contributing
a tensile force on the lower side of the stem.
