In order to reorientate stems and branches of (most) trees, reaction wood is
produced which provides a force in order to reorientate the tissue \cite{gardiner2014biology}.
Reorientation occurs for a number of recessions, it may be toward the light or upwards as is defined by the
negative gravitropism hypothesises or reducing wind drag \cite{niklas2012plant}\cite{coutts1995wind}. In softwoods this
reorientation is caused by the production of compression wood. Compression wood
forms on the outside of the stem or branch and expands longitudinally \cite{timell1986compression}. Hardwoods
on the other hand produce tension wood on the inside of the desired curve which
contracts longitudinally resulting in a curve forming \cite{gardiner2014biology}. Traditionally the
gelatinous layer (G-layer), a layer primarily consisting of low angle cellulose
fibrils on the inside of the fibres, is credited with forming growth stresses
within the tension wood. However some hardwoods produce tension wood
without producing a G-layer such as \textit{Eucalyptus Nitens} \cite{walker1993primary}.
