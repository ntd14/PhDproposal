koehler (1933) showed that a saw cut radially through a disk has a tendency to
close near the periphery suggesting that the peripheral cells are under
tangential compression with the inner cells under radial tension. He suggested this was the
cause of shakes in standing timber.
Jacobs 1945 removed inner circles from disks of a number of species and found
when an inner portion is removed the disks circumference increases. Jacobs again
argued that strain in the sap stream along with cells being wider tangentially
than radially led to the observed lateral stresses. Although he also mentions
the possibility of secondary thickening from the deposition of lignin as a
contributing factor.