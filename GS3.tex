After Martley's work a small number of authors investigated growth stresses
through the 30's and 40's. \cite{jacobs1945l}, testing 34 hardwood species, focused
mainly on Eucalyptus and argued that (longitudinal) tension successively
develops in the outer layers of the stem as it grows, and as a consequence of
the tension, compression must form in the centre of the stem. 
Experimentally Jacobs made use of strip planking, measuring the deflection of
the board after removal from the log, and the length change when the planks were
forced back straight. He showed that wood tends to shrink in the longitudinal
direction at the periphery while extend near the pith (indicating the log
is under compression in the centre and tension at the extremities).