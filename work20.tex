It is suspected that the most efficient way to minimise the issues growth
stresses cause during the production of timber is through appropriate genetic
selection. Eucalyptus species, in particular \textit{E. bosistoana} are showing
promise within the NZDFI trials to produce high value naturally durable
structural timber. In order to see the yield efficiency required to make this
product profitable, growth stresses need to be reduced to minimise the effects
discussed in section ---. While within the NZDFI project there are a number of
other concerns for breeders (such as durability, form and growth rate) growth
stresses also need to be considered. Using conventional breeding methods
discussed below, growth stresses will be minimised within the NZDFI genetics.
Currently two trials have been established or will soon be established, these
include: