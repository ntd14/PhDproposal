\section{Cells and wood in the context of a whole tree}
Growth stresses develop as part of cell formation and are thought to provide a
superior mechanical structure \cite{mattheck1997wood}. The continual formation of new cells contracting
on the periphery of the stem causes the older wood which has completed formation
to contract further with each new layer of cells. Older wood near the centre
of the stem becomes compressed while the newer cells cannot fully contract and
remain in tension \cite{Archer_1987}.
Growth stresses in normal wood increase the mechanical stability of the stem by increasing resistance to compression failure. In reaction wood growth stress provides the ability for the stem to reorient in
order to be best adapted to its environment.Growth stresses allow for an adaptive organism to survive in a changing environment,
however they also cause significant value loss when harvesting and
milling timber \cite{kubler_1987}.
