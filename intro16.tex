Growth stresses develop as part of cell formation and are thought to provide a
superior mechanical structure --ref--. The continual formation of new cells contracting
on the periphery of the stem causes the older wood which has completed formation
to contracted further with each new layer of cells. Older wood near the centre
of the stem becomes compressed while the newer cells can not fully contract and
remain in tension --ref--, until the bond between the old wood and new is separated
releasing the forces restricting this contraction (and extension in the centre).
Growth stresses in normal wood increase the mechanical stability of the stem --ref--,
while in reaction wood they provides the ability for the stem to reorient in
order to be best adapted to its environment at any given time --ref--. These properties
of wood allow for an adaptive organism to survive in a changing environment --ref--,
however they also cause significant issues and value loss when harvesting and
milling the timber they produce --ref--.
