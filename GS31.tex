\citet{Alm_ras_2005}
although similar made some major advancements over the \citet{Yamamoto_1998} model,
producing what is currently the most advanced mechanical model for growth
stress generation available. Previously fibres had been assumed to either be
free \citep{Yamamoto_1998} or fully restrained \citep{archer1987,archer1989}. Here
various boundary conditions are investigated, the most realistic arrangement being
displacement fully restrained in the longitudinal and tangential direction while
free in the radial. The virtually isolated fibre conditions were simulated
and found to be in good agreement, although with some small discrepancies from
\citet{Yamamoto_1998} (due to the introduction of some second order terms).  Their
investigation showed that differing boundary conditions had only a small effect
on the longitudinal strain, however the tangential strain was significantly
effected. This is explained as the cellulose is considered to already be stiff at
the start of maturation, therefore all of the stress within the cellulose can be
released as strain. However in the tangential direction the stiffness of the
fibre progressively increases as maturation proceeds, resulting in the
releasable strain being only a fraction of the total stress. In order to get good
experimental agreement \citet{Alm_ras_2005} used a transverse strain release
parameter allowing some strain to be released during maturation. They
found that 74\% of the transverse stress needs to be released during maturation
to provide the best agreement with experimental data.