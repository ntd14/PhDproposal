\chapter{Introduction}
As trees grow they produce wood in order to become taller, wider or orientate
stems, branches and roots. Becoming taller or reorienting stems and branches can
be an effective way to outcompete the other plants for light.
With increasing height, width or off axis stems comes increasing gravitational
force, wind drag and internal pressures (for water transport), which requires
either enough redundant strength in the existing structure (such as in
monocotyledons) or for the tree to strengthen its structure as it increases its
size. In wood plants size increase and reorientation occurs in
two ways, apical and cambial growth on branches, roots and the stem(s).