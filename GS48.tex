After the developments of Boyd and Jacobs in testing for growth stresses it
became apparent there was a need for a rapid testing procedure. \cite{Nicholson_1971}
developed the first of these measuring the released strain between two metal pins
on the surface of the sample, cut from the surface of logs. While considered a
rapid method in 1971, updated versions of this test are still used for measuring
surface strains but not practical (or considered rapid) for testing larger
numbers of stems such as in breeding trials. The `French` method (current
iteration \cite{Baill_res_1995}) involves drilling a hole between two reference
points, with a dial measuring the distance change between the two points.