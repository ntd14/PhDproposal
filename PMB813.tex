Due to the nature of the “Splitting” test, strains which result in the closure of the specimen cannot be measured, and as a result are recorded as zero. Fig. 2 shows a number of individuals exhibited closing, particularly when grown from seed, indicating an atypical stress pattern in the stem  \citep{Meinzer2011} with greater contraction at the pith than the periphery. The inverted stress-profile may be a result of tension wood being formed by the trees at young age (i.e. at the centre of the stem) to straighten the stem followed by normal wood with lower axial tensile growth-stresses at the periphery. Trees can form reaction wood in response to wind loading \citep{coutts1995wind}. A closing sample may indicate increased sensitivity to wind loading and the development of reaction wood at a young age in that genotype. In order to better understand the mechanisms for this unusual behaviour, a new trial has been established in which initial bending will be induced in the stems at a young age followed by straightening.