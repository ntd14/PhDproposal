The cell wall anatomy of different wood types (tension, normal and opposite)
needs to be investigated for the various NZDFI species (principally \textit{E.
bosistoana}). The anatomy study will consist of investigating which species
produce a G-layer (microscopy with staining) and the cell wall architecture
(Atomic Force and Electron Microscopy), cellulose, lignin and other constituents
volume fractions (Acid hydrolysis combined with NMR studies)  and the MFA and
the MFA standard deviation in all three wood types (x-ray diffraction). Fibre
diameter, length and lumen size will also be obtained (microscopy). Within
tension wood the removal of the G-layer (in G-layer producing species) will be
needed in order to determine the secondary cell wall properties of tension wood
(enzymatic removal).