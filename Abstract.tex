It has been shown previously that growth stresses are induced during secondary
cell wall formation, however the mechanism responsible is unknown. Two
hypotheses exist to explain the stress generation; 1) deposition of lignin
between the fibril aggregates, or 2) a mechanism resulting in
a length change of the cellulose itself causing the cell to expand or contract axially
depending on the orientation of the fibril (MFA). A combination of the two has also been proposed.

The proposed research will increase our understanding of the mechanisms behind
growth stress formation through the development of a mathematical model taking
into account a realistic cell wall supramolecular architecture to simulate cell wall 
maturation. The geometric parameters of the molecular cell wall architecture and
chemical composition will be studied using x-ray diffraction, atomic force
microscopy and wet chemistry. Experimentation will also be used to
investigate the role of the gelatinous (G) cell wall layer in the formation of tension wood.

The New Zealand Dryland Forest Initiative (NZDFI) is currently breeding
\textit{Eucalyptus bosistoana} as a high value timber crop alternative to
\textit{Pinus radiata}. One of the goals of the breeding trials is to select for
minimal growth stresses so that the timber does not loose value when harvesting
and milling due to the release of internal stresses. The proposed research will
select families at young ages which are producing the lowest possible growth
stresses improving the \textit{E. bosistoana} breeding stock.