\cite{Boyd_1972} presented (or rather popularised) the alternative (more widely
accepted) hypothesis of lignin swelling (first conceived by \cite{munch1938}, in German, reviewed in \cite{Boyd_1972}). Tensile
stress is gained in cells of low MFA by lignin deposition into the cell wall,
pushing the cellulose fibrils apart, which in turn shrinks the longitudinal
length of the cell and increases the tangential width. When MFA is high, the
opposite occurs, lengthening the cell and reducing its tangential width. This
shape change is not readily apparent in compression wood (characterised as short
tracheids) until the release of the stress acting on the CW, where by the cells
become longer and skinnier.
