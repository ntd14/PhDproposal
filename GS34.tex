\cite{walker2006primary} suggested that at the edge of the cellulose fibrils the cellulose becomes
disordered and is consequently able to bond with hemicelluloses, which have a
slightly shorter repeat length than the cellulose crystal. \cite{Davidson_2004} provided some evidence for the theory showing an increase in the fraction of interior chains resulted in an increase in repeat length. These hemicelluloses bonded to the outside of the fibril cause the fibril to be compressed in the
crystalline centre. An interesting consequence
is the contraction of the cellulose due to the hemicellulose bonding should be
dependent on the area/volume to circumference/surface area ratio as would be suggested by the results from \cite{Davidson_2004.