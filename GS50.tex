Measuring strains inside the stem proved to be more difficult. \cite{kikata1972}
adopted Jacob's planking method and electric strain gauges for improved accuracy (presented in \cite{kubler_1987} review).
\cite{wilhelmy1973probe} drilled holes of known diameters into stems and
attempted to measure the change in shape of the hole as the log was successively
cross cut closer to the test site, similar to \cite{boyd1950a}. \cite{ISI:A1979HU45700004}  attempted to measure the effect of growth stresses on increment cores.
They found that the stresses had an effect on the core itself
squashing it into an oval shape. \cite{FERRAND_1982} found a correlation between -0.67 and -0.77 for the relationship between
longitudinal strain and tangential core diameter. Showing
they can be used for near non-destructive growth strain testing.