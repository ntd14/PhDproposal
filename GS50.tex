Measuring strains inside the stem proved to be more difficult. Kikata (1972)
adopted Jacobs planking method and electric strain gauges for improved accuracy.
Kubler (1959) and Wilhelmy and Kubler (1973) drilled holes of known diameters into stems and
attempted to measure the change in shape of the hole as the log was successively
crosscut closer to the test site, as Boyd (1950) had done. Ploge and Thiercelin
1979 attempted to measure the effect of growth stresses on increment cores,
although they found that the stresses had an effect on the corer its self
squashing it into an oval shape. Ferrand 1982 found a correlation between
longitudinal strain and tangential core diameter between -0.67 and -0.77, showing
they can be used for near non destructive growth stress testing.