In order to select for low growth stress producing families experimental tests
need to be undertaken on each plant. The main test to be used to determine the
extent of growth stresses within the breeding population is the split test (also
known as the Pairing Test) as described in Chauhan (2010) The test involves
taking a significantly long section of stem and cutting along the pith to create
a radial split. The diameter of the stem is taken before testing and the width
of the opening measured immediately after splitting. Once the opening is
measured the stem is cut to across the grain to give two samples (one from each
side of the split). Density is measured by measuring the mass (using balances)
and volume using the displacement method on each of the pieces. Acoustics are
also taken using wood-spec to calculate the dynamic modulus, and hence the
stress can be derived Chauhan (2010). Due to the size of the Woodville trial it
may be the case that less tests are carried out. Decisions on the essential
tests will be made near the time of harvesting. Throughout testing the samples
are kept in a green state.