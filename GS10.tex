The Poisson effect is the revelation that a change of dimension of a material
in one direction will result in a change of dimension in the perpendicular directions.
This relationship is characterised by the Poisson ratio (within the elastic
region of deformation). Within growth stress literature there has been some
investigation of this effect as it appears the redistribution of stress through
the Poisson ratio from the longitudinal to tangential direction is not
sufficient to account for the observed tangential strains, which can also vary
for a given longitudinal strain. --ref--