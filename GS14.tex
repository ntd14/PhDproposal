Around the same time two other lesser known hypothesis were presented, \citet{hejnowicz1967some} argued that the stresses in compression
wood are related to the inhibition of water by the cell walls, which results in
swelling, because the expansion of compression wood is equal to the shrinkage
due to drying. \citet{brodzki1972} hypothesised strains due to 1,3-linked glucan (callase)
deposition within the helical checks of the S2 cell wall layer could be the
most significant factor in longitudinal growth stress generation. \citet{boyd1977basic}
refuted this idea arguing (along with other issues) that the callase would
expand into the cell lumen not causing any stresses in the cell wall, unless a
(non-observed) constraining median restricted the expansion.